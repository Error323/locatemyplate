%% vi: set tabstop=2, set textwidth=80
\documentclass[a4paper,11pt]{article}

\usepackage{homework}
\usepackage{graphicx}
\usepackage{verbatim}
\usepackage{algorithm}
\usepackage{algorithmic}
\usepackage{amsmath, amsthm, amssymb}
\usepackage[english]{babel}

\title{Locate My Plate \\ A License Plate Localisation System}

\date{June 24, 2009}

\begin{document}

\maketitle
\section*{Introduction}
This report describes the implementation of a robust, real-time License Plate
Localisation System (LPL). This system uses techniques discussed in
\cite{dlagnekov_thesis, zhang}. Using supervised learning, the system generates
a cascading classifier which consists of layers that each holds one strong
classifier. These strong classifiers are a linear function of several weak
classifiers. Each weak classifier is a feature which each describe
characteristics of a license plate. The first section describes the data used
for our experiments. The second section describes the features: What they are
and how they are generated.  Next the training and classification of weak-,
strong- and cascading classifiers is explained. Then the results are explained
and finally the conclusions.

\section*{Dataset}
The dataset used is obtained from \cite{dlagnekov_dataset}. It contains $246$ car
images with a resolution of approximatly $640\times480$ which were rescaled by
$50\%$ and annotated on location and size of the license plate. The dataset is
divided in a train-, test- and validation set with $159$, $40$ and $47$ images
respectively.


\section*{Features}
The core of the LPL system consists of features as described by
\cite{dlagnekov_thesis, zhang,naturaltext}. Features are image filters on a
certain type of image (e.g. an x-derivative). Formally a feature $f$ is a
tuple $\langle i, B, o \rangle$ defined as:
\begin{itemize}
	\item[$i$]{an index corresponding to the image type.}
	\item[$B$]{the set of blocks $b \in B$ where each block contains a sign
	$b_s \in \{-1,1\}$ which indicate subtraction or addition of that block.
	And relative positions $b_{pa},b_{pb} \in [0,1]$. which the determine the
	position of the block within the feature.}
	\item[$o$]{the orientation of the feature blocks: horizontal or vertical.}
\end{itemize}
The feature value $f:x\mapsto\mathbb{R}$ of an image $x$ is shown in Algorithm
\ref{alg:value}.
\begin{algorithm}
	\caption{featureValue($f$, $x$, $w$, $h$): Returns the image $V = f(x)$}
	\begin{algorithmic}[1]
	\REQUIRE The feature $f = \langle i, B, o \rangle$, the image $x$, the width $w$ and height $h$ of the feature.
	\medskip
	\STATE Let $V$ be an image with the same dimensions as $D(x)-[w, h]$ consisting of zeros.
	\STATE Let $I$ be the $i^{th}$ image type of $x$.
	\IF {$o$ is horizontal}
		\STATE $I \leftarrow I^\top$
		\STATE Swap $w$ and $h$
	\ENDIF
	\STATE Let $B'$ be the set of blocks with rescaled coordinates using $w,h$.
	\FORALL {$b \in B'$}
		\STATE Let $X$ be the result of applying $b$ to $I$ while respecting $b_{pa}, b_{pb}$
		\STATE $V \leftarrow V + b_s \cdot X$
	\ENDFOR
	\RETURN $V$
	\end{algorithmic}
\label{alg:value}
\end{algorithm}

\subsection*{Generation}
By representing a feature as a binary string, the set $S$ of possible features
can be easily calculated:
$$S = \{b(x) | \forall x \in \{1\ldots(x^n-2)/2\}\},$$
where $b(x)$ represents $x$ as a binary string and $n$ is the number of
segments. Each element in the binary string $s \in S$ represents the position
and the sign of a feature segment. Adjacent segments that share the same sign
are merged together and are called a feature block.

\subsection*{Optimization}
Features often share the same feature block dimensions. Even a feature
itself could have more feature blocks of the same dimensions. Because the
feature is applied as an image filter, a.
To optimize this, the feature blocks are individually calculated for every
position $(i,j)$ in the image and stored on its dimensions $(w,h)$ in a hash
table $R$.

\subsection*{Image types}
The features where applied to the following image types.
\begin{itemize}
	\item{1th order derivative in both x and y directions.}
	\item{2nd order derivative in both x and y directions.}
	\item{variance in both x and y directions.}
\end{itemize}
Before applying the feature, the above image types are passed through an
absolute filter. By using an integral image \cite{viola} per image type, the
featureblocks can be calculated very efficiently.


\section*{Training}
The overall cascading classifier consists of three types of training. The first
type is the training of the weak classifiers using features. The second type is
a linear combination of one or more weak classifiers into a strong classifier
using a boosting algorithm. The third type is a cascading classifier
with a strong classifier on each layer.

\subsection*{Weak classifier}
A weak classifier consists of a feature, a threshold $t \in \mathbb{R}$ and an
operator $\circ \in \{<, >\}$ which separates positive and negative samples as
good as possible according to the trainings set. After training, the weak
classifier $C$ constructs a binary image $B = t \circ f(x)$, where $x$ is the
image and $f$ the function that returns the value of the feature as described
in the Features section. The locations of the ones in $B$ correspond to the
location of possible license plates.

\subsection*{Strong classifier}
A strong classifier is constructed according to the boosting algorithm
described by \cite{viola}.  By re-weighting the positive and
negative samples after selecting a weak classifier, the algorithm selects the
"best" features with their respective weight or $\alpha$. Classification is
performed as follows:
\begin{displaymath}
C(x) = 
	\left\{ \begin{array}{ll}
		1 & \sum^T_{t=1} \alpha_t W_t(x) \ge \frac{1}{2} \sum^T_{t=1}\alpha_t \\
		0 & \textrm{otherwise}
	\end{array} \right.
\end{displaymath}
where $T$ is the number of weak classifiers and $W_t$ a function which returns
the binary image as described in the Weak classifier section.

\subsection*{Cascading classifier}
The cascading classifier is the final classifier. It consists of multiple
layers where each layer holds a strong classifier. This classifier is
trained as described by \cite{viola}. 
\begin{algorithm}
	\caption{cascadingClassify($C$, $x$, $w$, $h$): Returns the binary image $B$ of $x$}
	\begin{algorithmic}[1]
	\REQUIRE $C$ the cascading classifier, $x$ the image, $w,h$ the dimensions of the features
	\medskip
	\STATE Let $B$ be an image with the dimensions $D(x) - [w,h]$ consisting of ones.
	\FORALL {$c_s \in C$}
		\STATE $B' \leftarrow c_s(x)$
		\STATE $B \leftarrow B \land B'$
	\ENDFOR
	\RETURN $B$
	\end{algorithmic}
\label{alg:casc}
\end{algorithm}
Algorithm \ref{alg:casc} shows the classification of an image using a trained
cascading classifier. The strong classifier $c_s \in C$ classifies according to
the Strong classifier section.

\section*{Results}
% TODO: confusion matrix
% TODO: info about datasets, nr examples, train test etc.
% TODO: performance cascader displayed in graph (on every layer display the exclusions)
% TODO: best features on every layer
% TODO: (optional endresult, car with rectangle at license plate)
% TODO: (optionoal prob image of a good performing feature)
% TODO: 

\section*{Conclusions}
%TODO ispell

\renewcommand\bibname{References}
\bibliography{references}
\bibliographystyle{IEEEtran}
\end{document}
