%% vi: set tabstop=2, set textwidth=80
\documentclass[a4paper,11pt]{article}

\usepackage{homework}
\usepackage{graphicx}
\usepackage{verbatim}
\usepackage{algorithm}
\usepackage{algorithmic}
\usepackage{amsmath, amsthm, amssymb}
\usepackage[english]{babel}

\title{Locate My Plate \\ A License Plate Localisation System}

\date{June 24, 2009}

\begin{document}

\maketitle
\section*{Introduction}
This report describes the implementation of a robust, real-time License Plate
Localisation System (LPL). Using supervised learning, the system generates a
cascading classifier which consists of layers that each holds one strong
classifier. These strong classifiers are a linear function of several weak
classifiers. Each weak classifier is a feature which each describe
characteristics of a license plate. The first section describes the data used
for our experiments. The second section describes the features: What they are
and how they are generated.  Next the training and classification of weak,
strong and cascading classifiers is explained. The results come next and
finally the conclusions.

\section*{Dataset}
The dataset used for training, validating and testing the LPL system is obtained
from \cite{dlagnekov_dataset}. It contains 293 car images with a resolution of
$640\times480$ which were rescaled by $50\%$ and annotated on location and size of
the license plate. The dataset is divided in a train-, test- and validation set
with respectively 199,47 and 47 images.


\section*{Features}
The core of the LPL system consists of features. Features are image filters on
a certain type of image (e.g. an x-derivative). Formally a feature $f$ is a
tuple $\langle i, B, o \rangle$ defined as:
\begin{itemize}
	\item[$i$]{an index corresponding to the integral image type.}
	\item[$B$]{the set of blocks $b \in B$ where each block contains a sign
	$b_s$ which indicate subtraction or addition of that block. And relative
	coordinates $x0,y0,x1,y1$ which the determine the location of the block
	within the feature.}
	\item[$o$]{which indicates the orientation: horizontal or vertical.}
\end{itemize}
The feature value $f:x\mapsto\mathbb{R}$ of an image $x$ is shown in Algorithm
\ref{alg:value}.
\begin{algorithm}
	\caption{featureValue($f$, $x$): Returns the real valued image $V$ of $x$ according to feature $f$}
	\begin{algorithmic}[1]
	\REQUIRE The feature $f = \langle i, B, o \rangle$, the image $x$
	\medskip
	\STATE Let $V$ be an image with the same dimensions as $x$ consisting of zeros.
	\STATE Let $I$ be the $i^{th}$ integral image of $x$.
	\IF {$o$ is horizontal}
		\STATE Let $I$ be $I'$ ($I$ transposed)
	\ENDIF
	\FORALL {$b \in B$}
		\STATE $V \leftarrow V + b_s \cdot b(I)$
	\ENDFOR
	\RETURN $V$
	\end{algorithmic}
\label{alg:value}
\end{algorithm}


\subsection*{Generation}
A feature is represented as a binary string. Each element in the binary string
the position and the sign of a feature segment. Adjacent segments that share
the same sign are merged together and are called a feature block. This
representation makes it easy to generate all possible features.

\subsection*{Optimization}
Lots of features share the same feature block dimensions. Even a feature
itself could have more feature blocks of the same dimensions. Because the
feature is used as a scanning window, lots of redundant calculations are done.
To optimize this, the feature blocks are individually calculated for every
position $(i,j)$ in the image and stored on its dimensions $(h,w)$ in a hash
table $R$.

\subsection*{Image types}
The features where applied to the following image types.
\begin{itemize}
	\item{1th order derivative in both x and y directions.}
	\item{2nd order derivative in both x and y directions.}
	\item{variance in both x and y directions.}
\end{itemize}
Before applying the feature, the above image types are passed through an
absolute filter.


\section*{Training}
The overall cascading classifier consists of three types of training. The first
type is the training of the weak classifiers using features. The second type is
a linear combination of one or more weak classifiers into a strong classifier
using a boosting algorithm. Finally the third type is a cascading classifier
with a strong classifier on each layer.

\subsection*{Weak classifier}
A weak classifier is a feature with a threshold $t \in \mathbb{R}$ and an
operator $\circ \in \{<, >\}$ which separates positive and negative samples as
good as possible according to the trainings set. After training, the weak
classifier $C$ constructs a binary image $B = t \circ f(x)$, where $x$ is the
image and $f$ the function of the feature as described in the Features section.
This binary image $B$ contains possible license plates according to the
feature.

\subsection*{Strong classifier}
A strong classifier is constructed according to the boosting algorithm
described by Viola and Jones \cite{viola}.  By re-weighting the positive and
negative samples after selecting a weakClassifier, the algorithm selects the
"best" features with their respective alpha. Classification is performed as
follows:
\begin{displaymath}
C(x) = 
	\left\{ \begin{array}{ll}
		1 & \sum^T_{t=1} \alpha_t W_t(x) \ge \frac{1}{2} \sum^T_{t=1}\alpha_t \\
		0 & \textrm{otherwise}
	\end{array} \right.
\end{displaymath}
where $T$ is the number of weak classifiers and $W_t$ a function which returns
the binary image as described in the Weak classifier section.

\subsection*{Cascading classifier}
The cascading classifier is the final classifier. It consists of multiple
layers where each layer represents a strong classifier. This classifier is
trained as described by Viola and Jones \cite{viola}. 
\begin{algorithm}
	\caption{classify($C$, $x$): Returns the binary image $B$ of $x$}
	\begin{algorithmic}[1]
	\REQUIRE $C$ the cascading classifier, $x$ the image
	\medskip
	\STATE Let $B$ be an image with the same dimensions as $x$ consisting of ones.
	\FORALL {$c_s \in C$}
		\STATE $B' \leftarrow c_s(x)$
		\STATE $B \leftarrow B \land B'$
	\ENDFOR
	\RETURN $B$
	\end{algorithmic}
\label{alg:casc}
\end{algorithm}
Algorithm \ref{alg:casc} shows the classification of an image using a trained
cascading classifier. The strong classifier $c_s \in C$ classifies according to
the Strong classifier section described above.

\section*{Results}
We used segments

confusion matrix
info about datasets, nr examples, train test etc.
performance cascader displayed in graph (on every layer display the exclusions)
best features on every layer
(optional endresult, car with rectangle at license plate)
(optionoal prob image of a good performing feature)


\section*{Conclusions}
TODO ispell

\renewcommand\bibname{References}
\bibliography{references}
\bibliographystyle{IEEEtran}
\end{document}
