%% vi: set tabstop=2, set textwidth=80
\documentclass[a4paper,11pt]{article}

\usepackage{homework}
\usepackage{graphicx}
\usepackage{verbatim}
\usepackage{algorithm}
\usepackage{algorithmic}
\usepackage{amsmath, amsthm, amssymb}
\usepackage[english]{babel}

\title{Locate My Plate \\ A License Plate Recognition System}

\date{June 24, 2009}

\begin{document}

\maketitle
\section*{Introduction}
This report describes the implementation of a robust, real-time License Plate
Recognition System (LPR). Using supervised learning, the system generates a
cascading classifier which consists of multiple layers that each holds one
strong classifier. These strong classifiers are a linear function of several
weak classifiers aka features which each describe characteristics of a license
plate. The first section describes the implementation of the algorithms
involved, next we show our results using an american dataset and finally we
conclude.


\section{integral images}
how is a integral image computed

\section{image descriptors}
dx, abs ddy, etc.
pictures

\section{features}
what is a feature..

	\subsection{feature generation}
	A feature can be defined as a scanning window contaning a set of ascending
	segments. The segments which form a group are the feature blocks. Each
	feature block takes the sum of all pixels the block TODO overlaps TODO (see
	integral images). Dependend of the sign of the individual featureblocks he
	is summed up or substracted. 

	Because the segments and featureblocks are vertically TODO gescheiden, the
	feature is called a vertical feature.  The horizontal features are simply
	generated using the transpose of this vertical feature.

	\subsection{binary feature generation}
	A feature can be represented as a binary code. Each element in the binary
	represents the position and the sign of a segment of a feature.
	example TODO picture\\

	This representation makes it easy to generate all possible features
	permutations using the powerset.  Because half of the features are simply
	the inverse of another feature the set is pruned.

	\subsection{optimalisation}
	optimalisation 1
	the use of matrices 

	optimalisation 2
	Because lots of featureblocks have the same dimension

	Lots of features share the same featureblock dimensions. Even a feature
	itself could have more featureblocks of the same dimensions.  Because the
	feature is used as a scanning window lots of unnecessary redundant
	calculations are done.

	To optimize this, the featureblocks are individually calculated for every
	position in the image and stored on its dimensions in a hash table.

	optimalisation 4
	shift matrices and add together


\section*{Implementation}

\section*{Results}

\section*{Conclusions}

\end{document}
