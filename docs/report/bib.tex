@article{zhang2006lbl,
	author = {Zhang, H. and Jia, W. and He, X. and Wu, Q.},
	citeulike-article-id = {1751300},
	comment = {The authors combine two kinds of features for licence plate detection, i.e. global statistics of scanning window and local haar features. Global statistics are gradient density (LP have higher density of edges) and density variance. The global features are at the beginning of the cascade which might be very time consuming, the authors say it is fast!},
	journal = {Proceedings International Conference on Pattern Recognition, page to appear},
	keywords = {density, detection, features, global, gradient, licence, lpr, plate},
	posted-at = {2007-10-10 17:30:38},
	priority = {0},
	title = {2006, Learning-Based License Plate Detection Using Global and Local Features},
	year = {2006}
}
@article{citeulike:942195,
	abstract = {This paper describes a face detection framework that is capable of processing images extremely rapidly while achieving high detection rates. There are three key contributions. The first is the introduction of a new image representation called the ldquoIntegral Imagerdquo which allows the features used by our detector to be computed very quickly. The second is a simple and efficient classifier which is built using the AdaBoost learning algorithm (Freund and Schapire, 1995) to select a small number of critical visual features from a very large set of potential features. The third contribution is a method for combining classifiers in a ldquocascaderdquo which allows background regions of the image to be quickly discarded while spending more computation on promising face-like regions. A set of experiments in the domain of face detection is presented. The system yields face detection performance comparable to the best previous systems (Sung and Poggio, 1998; Rowley et al., 1998; Schneiderman and Kanade, 2000; Roth et al., 2000). Implemented on a conventional desktop, face detection proceeds at 15 frames per second.},
	address = {Hingham, MA, USA},
	author = {Viola, Paul and Jones, Michael J.},
	citeulike-article-id = {942195},
	doi = {10.1023/B:VISI.0000013087.49260.fb},
	issn = {0920-5691},
	journal = {International Journal of Computer Vision},
	keywords = {boosting, face\_detection},
	month = {May},
	number = {2},
	pages = {137--154},
	posted-at = {2008-07-23 16:24:33},
	priority = {2},
	publisher = {Kluwer Academic Publishers},
	title = {Robust Real-Time Face Detection},
	url = {http://dx.doi.org/10.1023/B:VISI.0000013087.49260.fb},
	volume = {57},
	year = {2004}
}
@TECHREPORT { Dlagnekov05,
    AUTHOR = { Louka Dlagnekov and Serge Belongie },
    TITLE = { Recognizing Cars },
    INSTITUTION = { University of California, San Diego },
    NUMBER = { UCSD CSE Tech Report CS2005-0833 },
    YEAR = { 2005 },
    PDF = { /Recognizing_Cars_Dlagnekov_Belongie.pdf },
}
@MASTERSTHESIS { Dlagnekov_thesis,
    TITLE = { Video-based Car Surveillance: License Plate, Make, and Model Recognition },
    AUTHOR = { Louka Dlagnekov },
    SCHOOL = { University of California, San Diego },
    YEAR = { 2005 },
}
@inproceedings{citeulike:3812995,
	abstract = {This paper gives an algorithm for detecting and reading text in natural images. The algorithm is intended for use by blind and visually impaired subjects walking through city scenes. We first obtain a dataset of city images taken by blind and normally sighted subjects. From this dataset, we manually label and extract the text regions. Next we perform statistical analysis of the text regions to determine which image features are reliable indicators of text and have low entropy (i.e. feature response is similar for all text images). We obtain weak classifiers by using joint probabilities for feature responses on and off text. These weak classifiers are used as input to an AdaBoost machine learning algorithm to train a strong classifier. In practice, we trained a cascade with 4 strong classifiers containing 79 features. An adaptive binarization and extension algorithm is applied to those regions selected by the cascade classifier. Commercial OCR software is used to read the text or reject it as a non-text region. The overall algorithm has a success rate of over 90\% (evaluated by complete detection and reading of the text) on the test set and the unread text is typically small and distant from the viewer.},
	author = {Chen, Xiangrong and Yuille, A. L.},
	booktitle = {Computer Vision and Pattern Recognition, 2004. CVPR 2004. Proceedings of the 2004 IEEE Computer Society Conference on},
	citeulike-article-id = {3812995},
	comment = {* text detection using AdaBoost on own feature set. Commercial OCR packages for text recognition
* ABBYY FineReader, TOCR, ReadIris Pro 8},
	doi = {10.1109/CVPR.2004.1315187},
	journal = {Computer Vision and Pattern Recognition, 2004. CVPR 2004. Proceedings of the 2004 IEEE Computer Society Conference on},
	keywords = {ocr},
	pages = {II-366--II-373 Vol.2},
	posted-at = {2008-12-20 04:11:27},
	priority = {2},
	title = {Detecting and reading text in natural scenes},
	url = {http://dx.doi.org/10.1109/CVPR.2004.1315187},
	volume = {2},
	year = {2004}
}

	


% Adaboost from wikipedia
